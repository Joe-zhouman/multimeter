\chapter{系统操作}
\section{系统起动}
应给出系统起动的详细过程。
\section{各种操作、命令和语言}
软件系统的使用过程都是使用软件系统提供的各种操作、命
令和语言的过程。
操作和命令:在用户手册中详细给出各种操作的过程和功能、命令的格式和功能;应
当描述在使用上的各种限制,如,操作状态、操作条件、操作序列等。另外,必要时
可以通过适当的举例讲述各种操作和命令的使用方法,以帮助用户理解。
输出信息:应该详细列出与操作、命令相关的各种输出信息。如果输出信息的意思本
身不是很明显,应当给予解释。另外还应当说明对于这些信息所采取的操作
程序设计语言〖条件〗:如果我们的软件系统提供了某种语言,对其语言规则应当给
予说明。关于程序设计语言的用户手册的内容,其详细说明我们以后补充。
\section{各种数据}在软件的使用过程中,用户必须与各种数据和信息打交道。为了让用户能
够操作我们的软件,我们必须为用户提供各种结构以及每个数据元素的含义。
有些数据适合在系统操作说明中给出,有些适合在后面的附录中给出,甚至有些除了在
操作说明的同时给出外,还要在附录中给予归纳,这些都由用户手册编写人员根据实际
兄来决定。这些数据包括
输出数据:应当给出软件以何种形式输出的数据的内容和格式,并要求以例样的形式
给予说明。
中间数据〖条件〗:如果我们告诉用户在软件的运行过程中所产生的中间数据的内容
和格式,有助于用户理解软件的使用,则应当给予说明。
数据限制〖条件〗:如果对数据有限制,如数据的大小限制,则应当给予说明。
数据文件〖条件〗:如果要告诉用户我们的软件所使用的某些数据文件的结构有助于
用户理解我们软件的使用,则应给予说明,但应该注意技术保密。如果对数据文件有
所限制,例如每个文件的最大记录数、每个磁盘的最大文件数等,应当给予说明
\section{处理过程}
如果我们简要地给用户描述我们软件对用户的操作、输入的命令
和输入数据的处理过程,有助于用户了解我们软件的使用,则应给予说明
\section{出错处理}
应当给出各种出错情况以及相应的处理措施
\section{操作技术}
有些软件的操作可能需要一定的技术和经验,才能获得满意的结
果,那么应该在用户手册上尽量给出这些技术和经验的描述,或告诉用户如何才能获得
这些技术和经验。例如,在操作SEAS系统作图纸净化处理时,如何选择适当的阀值就是
需要一定的技术和经验的问题。